\documentclass[12pt, a4paper]{article}
\usepackage{graphicx} 
\usepackage{geometry}
\geometry{a4paper, margin=1in}
\usepackage{tikz}
\usetikzlibrary{calc}
\usepackage{listings}
\usepackage{xcolor}

\lstset{ 
    language=C,                     % Language of the code
    basicstyle=\ttfamily\footnotesize, % Code font and size
    keywordstyle=\color{blue},      % Keywords color
    stringstyle=\color{red},        % String color
    commentstyle=\color{green},     % Comment color
    morecomment=[l][\color{magenta}]{\#}, % Define more comment style
    numbers=left,                   % Line numbers on the left
    numberstyle=\tiny\color{gray},  % Line numbers style
    stepnumber=1,                   % Line numbers step
    numbersep=10pt,                 % Line numbers separation
    tabsize=4,                      % Tab size
    showspaces=false,               % Do not show spaces
    showstringspaces=false,         % Do not show string spaces
    breaklines=true,                % Automatic line breaking
    breakatwhitespace=false,        % Break at whitespace
    frame=single,                   % Frame around the code
    title=\lstname,                 % Show file name
    escapeinside={\%*}{*)},         % Escape inside code
    morekeywords={*,...}            % Custom keywords
}



\title{\Huge \textbf{Software Tools And Technology}} 
\date{}
\author{Group 29}

\begin{document}

\begin{figure}
    \centering
    \includegraphics[width=0.3\linewidth]{makaut.png}
\end{figure}

\maketitle
\pagenumbering{gobble}

\begin{tikzpicture}
[remember picture, overlay] \draw[line width = 2pt] ($(current page.north west) + (0.5in, -0.5in)$) rectangle ($(current page.south east) + (-0.5in, 0.5in)$);
\end{tikzpicture}

\begin{center}

\LARGE\textbf{Lab Notebook}

\end{center}

\centering
\vspace{0.5cm}

\bfseries{\underline{Group members:}}

\begin{enumerate}
        \item Falguni Chakraborty Bsc in IT(Data science)
        \item Sounak Kundu Bsc in IT(AI)
        \item Udita Sarkar BCA
        \item Priyangshu Paul BCA
        \item Mahwish Gul BCA
\end{enumerate}

\vspace{2cm}
 \textbf{Instructor:} Ayan Ghosh \\
 \textbf{Course:} Software Tools And Technology

\newpage

\usetikzlibrary{calc}

\begin{tikzpicture}
[remember picture, overlay] \draw[line width = 2pt] ($(current page.north west) + (0.5in, -0.5in)$) rectangle ($(current page.south east) + (-0.5in, 0.5in)$);
\end{tikzpicture}

\centering

\begin{center}
\Large{\bfseries\underline{Lab Notebook Entries}}
\end{center}

\section{Lab Entry by Falguni Chakraborty}
Reg no- 233002410601
Roll no- 30084323002

\subsection{Experiment}


\begin{table}[ht]
\centering
\begin{tabular}{|p{50pt}|p{200pt}|}
\hline
\textbf{Sl. No.} & \textbf{Assignments} \\ \hline
1. & Introduction to Github and Github desktop version installation \\ \hline
\end{tabular}
\end{table}

\section{Lab Entry by Sounak Kundu}
Reg no- 233002410576
Roll no- 30054623024
\subsection{Experiment}


\begin{table}[ht]
\centering
\begin{tabular}{|p{50pt}|p{200pt}|}
\hline
\textbf{Sl. No.} & \textbf{Assignments} \\ \hline
1. & Building a C programme of calculator in the local repository, committing and publishing it as a public repository. \\ \hline
\end{tabular}
\end{table}

\section{Lab Entry by Mahwish Gul}
Reg no- 233001010529
Roll no - 30001223055
\subsection{Experiment}

\vspace{0.5cm}
\begin{table}[ht]
\centering
\begin{tabular}{|p{50pt}|p{200pt}|}
\hline
\textbf{Sl. No.} & \textbf{Assignments} \\ \hline
1. & Converting submit button to Chin tapak dum dum \\ \hline
\end{tabular}
\end{table}

\newpage

\usetikzlibrary{calc}

\begin{tikzpicture}
[remember picture, overlay] \draw[line width = 2pt] ($(current page.north west) + (0.5in, -0.5in)$) rectangle ($(current page.south east) + (-0.5in, 0.5in)$);
\end{tikzpicture}
\section{Lab Entry by Udita Sarkar}
Reg no: 233001010477
Roll no: 30001223003
\subsection{Experiment}

\vspace{0.5cm}
\begin{table}[ht]
\centering
\begin{tabular}{|p{50pt}|p{200pt}|}
\hline
\textbf{Sl. No.} & \textbf{Assignments} \\ \hline
1. & Introduction to LaTeX \\ \hline
\end{tabular}
\end{table}

\vspace{2.5cm}
\section{Lab Entry by Priyangshu Paul}
Reg no :- 233001010505
Roll No :- 30001223031
\subsection{Experiment}

\vspace{0.5cm}
\begin{table}[ht]
\centering
\begin{tabular}{|p{50pt}|p{200pt}|}
\hline
\textbf{Sl. No.} & \textbf{Assignments} \\ \hline
1. & Creating LaTeX repository on github \\ \hline
\end{tabular}
\end{table}

\newpage

\usetikzlibrary{calc}

\begin{tikzpicture}
[remember picture, overlay] \draw[line width = 2pt] ($(current page.north west) + (0.5in, -0.5in)$) rectangle ($(current page.south east) + (-0.5in, 0.5in)$);
\end{tikzpicture}

\section*{Github}
\paragraph{GitHub is a web-based platform that allows developers to host, share, and collaborate on software projects. It provides a version control system powered by Git, enabling teams to track changes, manage code repositories, and work together seamlessly, even across different locations. GitHub supports collaborative development through features like pull requests, issues, and project boards, making it an essential tool for open-source projects and professional software development alike. Additionally, it offers integration with various development tools, enhancing productivity and streamlining the software development lifecycle.}

\begin{figure}
    \centering
    \includegraphics[width=0.5\linewidth]{GitHub-Symbol.png}
\end{figure}
\subsection*{Installation}
\paragraph{Installing GitHub Desktop is a straightforward process that enhances your workflow by providing a user-friendly interface for managing repositories. To begin, download the installer from the [official GitHub Desktop website](https://desktop.github.com/) for your operating system—Windows or macOS. After downloading, simply run the installer and follow the on-screen instructions to complete the setup. Once installed, you can launch the application and sign in with your GitHub credentials, or create a new account if needed. GitHub Desktop streamlines the process of cloning repositories, making commits, and managing branches, making it an invaluable tool for developers of all skill levels. For Linux users, alternative methods like using Wine or other Git clients are available.}

\newpage

\begin{tikzpicture}
[remember picture, overlay] \draw[line width = 2pt] ($(current page.north west) + (0.5in, -0.5in)$) rectangle ($(current page.south east) + (-0.5in, 0.5in)$);
\end{tikzpicture}

\section*{Building a C programme of calculator in the local repository,committing and publishing it as a public repository.}
\vspace{1cm}
Building a C program for a calculator and managing it with GitHub Desktop involves a few straightforward steps. First, write the C program that performs basic arithmetic operations like addition, subtraction, multiplication, and division. Once your program is complete, you can manage and publish it using GitHub Desktop.
\vspace{1.5cm}
\textbf{1. Set Up Your Project Locally:} Create a folder on your computer to store your C program files. Place all the related files inside this folder.
\vspace{1.5cm}
\textbf{2.Initialize a Repository in GitHub Desktop:}
    - Open GitHub Desktop and go to "File" > "New Repository".
    - Choose the local folder where your C program is stored.
    - Name your repository, add a description if needed, and choose whether to initialize it with a README file.
\vspace{1.5cm}
\textbf{3.Commit Your Code:}
    - In GitHub Desktop, you’ll see any uncommitted changes in your repository. Write a commit message that describes your changes (e.g., "Initial commit of calculator program") and click "Commit to main" (or the appropriate branch name).
\vspace{1.5cm}
\textbf{4.Publish Your Repository:}
    - After committing your changes, click "Publish repository" in the top bar.
    - Choose whether you want the repository to be public or private, then click "Publish repository".
\vspace{1.5cm}
Your C program is now available on GitHub as a public repository, making it accessible to others who can view, download, or contribute to your project. This process allows you to manage your code using a graphical interface without needing to use Git commands in the terminal.

\newpage

\begin{tikzpicture}
[remember picture, overlay] \draw[line width = 2pt] ($(current page.north west) + (0.5in, -0.5in)$) rectangle ($(current page.south east) + (-0.5in, 0.5in)$);
\end{tikzpicture}


\title{C Program for a Calculator}
\author{Your Name}
\date{\today}
\maketitle

\section{C Code}
\begin{lstlisting}
#include <stdio.h>

int main() {
    char operator;
    double num1, num2, result;

    // Prompt user for the operation
    printf("Enter an operator (+, -, *, /): ");
    scanf("%c", &operator);

    // Prompt user for the numbers
    printf("Enter two operands: ");
    scanf("%lf %lf", &num1, &num2);

    // Perform the chosen operation
    switch (operator) {
        case '+':
            result = num1 + num2;
            printf("%.2lf + %.2lf = %.2lf\n", num1, num2, result);
            break;
        case '-':
            result = num1 - num2;
            printf("%.2lf - %.2lf = %.2lf\n", num1, num2, result);
            break;
        case '*':
            result = num1 * num2;
            printf("%.2lf * %.2lf = %.2lf\n", num1, num2, result);
            break;
        case '/':
            if (num2 != 0) {
                result = num1 / num2;
                printf("%.2lf / %.2lf = %.2lf\n", num1, num2, result);
            } else {
                printf("Error! Division by zero is not allowed.\n");
            }
            break;
        default:
            printf("Error! Operator is not correct\n");
    }

    return 0;
}
\end{lstlisting}
\newpage

\begin{tikzpicture}
[remember picture, overlay] \draw[line width = 2pt] ($(current page.north west) + (0.5in, -0.5in)$) rectangle ($(current page.south east) + (-0.5in, 0.5in)$);
\end{tikzpicture}

\section*{Java Swing Application: SymbolApp}
This document describes the Java Swing application named \texttt{SymbolApp}. The application showcases a simple "mind-reading" trick by displaying a grid of symbols and revealing a selected symbol based on user interaction. This document is formatted using LaTeX to provide a clear and professional presentation for academic purposes.

\section*{ 1. Clone the Repository}
At first, I used GitHub Desktop to clone the repository: https://github.com/GeekAyan/STT.
Then I Opened GitHub Desktop, clicked on "File" then "Clone Repository", pasted the URL, and selected my local directory.

\section*{  2. Set Up the Project}
I opened the project in VSCode.
Then Followed the detailed run instructions provided in the README.md file to set up any necessary dependencies and configurations. This could involve installing Python packages or setting environment variables.

\section*{ 3. Run the Application}
I Run the application according to the instructions to ensure everything is working as expected.

\section*{ 4. Modify the Button}
I located the code for the button in the project files. This could be in a JavaScript, HTML, or Python file, depending on the technology stack used.
I Renamed the button text to \textbf{Chin Tapak Dum Dum"}.
  
\section*{ 5. Fix the Button Proportions}
After renaming the button, I analyzed why the button looks disproportionate. Possible fixes could involve. I adjusted something and modified the code.

\section*{ 6. Test the Changes}
I ran the application again to ensure the button now appears correctly proportioned and that it functioned as intended.

\section*{ 7. Commit the Changes}
I saved all the changes in your IDE.
In GitHub Desktop,I commited the changes with a descriptive message like “Fixed button proportions and renamed to 'Chin Tapak Dum Dum'”.

\section*{ 8. Push Changes to my Fork}
I pushed the changes to my forked version first.

\section*{9. Create a Pull Request}
I went to the original GitHub repository on my web browser.
Clicked on “Pull Requests” then “New Pull Request”.
Compare my branch with the main branch of the original repository.
Added a title and description explaining my changes and why they were made.

\begin{tikzpicture}
[remember picture, overlay] \draw[line width = 2pt] ($(current page.north west) + (0.5in, -0.5in)$) rectangle ($(current page.south east) + (-0.5in, 0.5in)$);
\end{tikzpicture}

\section*{Code Description}
The \texttt{SymbolApp} class extends \texttt{Frame} and implements \texttt{ActionListener}. It generates a random symbol and displays it among other symbols in a grid layout. The user follows specific steps to select a symbol, which is then revealed when the submit button is clicked.

\subsection{Code Listing}
\begin{lstlisting}[language=Java, caption=Java Swing Application Code]
import java.awt.*;
import java.awt.event.*;
import java.util.Random;

public class SymbolApp extends Frame implements ActionListener {
    private Label[] symbolLabels = new Label[99];
    private Button submitButton;
    private String specialSymbol;
    private String selectedSymbol;

    public SymbolApp() {
        // Generate a random special symbol
        Random rand = new Random();
        specialSymbol = Character.toString((char) (rand.nextInt(94)
        + 33)); // Random ASCII character from 33 to 126
        selectedSymbol = specialSymbol;

        // Setting up the main frame
        setLayout(new BorderLayout());
        setSize(800, 700);
        setTitle("Symbol App");

        // Adding instruction message
        TextArea instruction = new TextArea(
            "Think of any two digit number. Now reverse it and find
            the difference of them.\n" +
            "Now find the number you got and remember the symbol from
            the panel below.\n" +
            "Don't tell me, I'll read your mind!
            Hit the below button when you are ready
            to see the magic!",
            5, 60, TextArea.SCROLLBARS_NONE);
        instruction.setEditable(false);
        instruction.setFont(new Font("Arial", Font.PLAIN, 16));
        add(instruction, BorderLayout.NORTH);

        // Panel for symbols
        Panel symbolPanel = new Panel(new GridLayout(11, 9));
        for (int i = 0; i < 99; i++) {
            String symbol = (i % 9 == 0) ? specialSymbol : 
            Character.toString
            ((char) 
            (33 + (i % 94)));
            symbolLabels[i] = new Label(i + ": " + symbol);
            symbolLabels[i].setAlignment(Label.CENTER);
            symbolPanel.add(symbolLabels[i]);
        }
        add(symbolPanel, BorderLayout.CENTER);

        // Panel for submit button
        Panel controlPanel = new Panel(new FlowLayout());
        // changed button name to chin tapak dum dum
        submitButton = new Button("Chin Tapak Dum Dum"); 
        submitButton.addActionListener(this);
        controlPanel.add(submitButton);
        add(controlPanel, BorderLayout.SOUTH);

        // Setting up the window close event
        addWindowListener(new WindowAdapter() {
            public void windowClosing(WindowEvent we) {
                System.exit(0);
            }
        });

        setVisible(true);
    }
\end{lstlisting}

\newpage 
\begin{tikzpicture}
[remember picture, overlay] \draw[line width = 2pt] ($(current page.north west) + (0.5in, -0.5in)$) rectangle ($(current page.south east) + (-0.5in, 0.5in)$);
\end{tikzpicture}

\section*{Explanation of Changes}
In this document, the submit button's text has been changed to \texttt{"Chin Tapak Dum Dum"} to enhance its visual appeal and engagement.
\section*{Conclusion}
The \texttt{SymbolApp} Java Swing application demonstrates basic GUI components and event handling. The use of LaTeX provides a clean and professional presentation for documenting the code and its functionality.\\
\newpage

\begin{tikzpicture}
[remember picture, overlay] \draw[line width = 2pt] ($(current page.north west) + (0.5in, -0.5in)$) rectangle ($(current page.south east) + (-0.5in, 0.5in)$);
\end{tikzpicture}

\centering
\section*{Introduction to LaTeX}
\vspace{2cm}
LaTeX, created by Leslie Lamport in the 1980s, is built on top of the TeX typesetting system developed by Donald Knuth. While TeX provides the foundational mechanics for document preparation, LaTeX simplifies the process by offering a higher-level language that automates many of the intricate formatting tasks. This separation of content from style allows authors to maintain consistency across documents without getting bogged down by the minutiae of layout adjustments. \\
\vspace{0.3cm}
One of LaTeX's standout features is its ability to handle large documents with complex structures, such as books, theses, or multi-author projects. It supports a wide range of document elements, including tables, figures, and footnotes, all of which can be managed with ease. Moreover, LaTeX's robust referencing system makes it straightforward to cite sources and manage bibliographies, which is essential in academic and scientific writing.\\
\vspace{0.3cm}
Another advantage of LaTeX is its cross-platform compatibility. Since LaTeX files are plain text, they can be edited on any operating system and easily shared across different environments. Additionally, LaTeX documents are typically compiled into PDF format, ensuring that the final output looks the same regardless of the viewer's software or device.\\
\vspace{0.3cm}
Though LaTeX may seem intimidating at first, a strong community of users and extensive documentation make it accessible to newcomers. With practice, users can harness LaTeX's capabilities to create polished, professional-quality documents with a level of control and precision that is hard to achieve with standard word processors. Whether you're writing a simple article or a comprehensive dissertation, LaTeX offers the tools you need to present your work effectively and elegantly.

\begin{figure}
    \centering
    \includegraphics[width=0.3\linewidth]{latex.png}
\end{figure}
@Me tor code

\newpage

\begin{tikzpicture}
[remember picture, overlay] \draw[line width = 2pt] ($(current page.north west) + (0.5in, -0.5in)$) rectangle ($(current page.south east) + (-0.5in, 0.5in)$);
\end{tikzpicture}

\centering
\section*{Creating LaTeX Repository on Github}
\vspace{2cm}
Creating a LaTeX repository on GitHub is a practical way to manage and collaborate on LaTeX projects, especially for academic or technical documents. GitHub provides a version control system using Git, which allows multiple contributors to work on the same document, track changes, and manage different versions of the project efficiently. Here’s a step-by-step guide to setting up a LaTeX repository on GitHub:\\

\vspace{2cm}

\textbf{1. Set Up a GitHub Account}\\
\vspace{0.5cm}
If you don’t already have one, start by creating an account on GitHub. This will be the platform where you host your LaTeX repository. \\

\vspace{2cm}

\textbf{2. Create a New Repository}\\
\vspace{0.5cm}
On GitHub, click on the "New" button in the "Repositories" section of your profile. Give your repository a name (e.g., "my-latex-project") and a description. You can choose to make it public or private, depending on whether you want to share your work with others. Optionally, add a README file to introduce your project and a .gitignore file configured for LaTeX to prevent unnecessary files from being tracked \\

\vspace{2cm}

By creating a LaTeX repository on GitHub, you benefit from a structured, version-controlled environment that facilitates collaboration, backup, and management of your LaTeX projects, whether you're working alone or as part of a team
\end{document}
