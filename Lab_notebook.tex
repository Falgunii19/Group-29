\documentclass[12pt, a4paper]{article}
\usepackage{graphicx} 
\usepackage{geometry}
\geometry{a4paper, margin=1in}
\usepackage{tikz}
\usetikzlibrary{calc}
\title{\Huge \textbf{Software Tools And Technology}} 
\date{}
\author{Group 29}

\begin{document}

\begin{figure}
    \centering
    \includegraphics[width=0.3\linewidth]{makaut.png}
\end{figure}

\maketitle
\pagenumbering{gobble}

\begin{tikzpicture}
[remember picture, overlay] \draw[line width = 2pt] ($(current page.north west) + (0.5in, -0.5in)$) rectangle ($(current page.south east) + (-0.5in, 0.5in)$);
\end{tikzpicture}

\begin{center}

\LARGE\textbf{Lab Notebook}

\end{center}

\centering
\vspace{0.5cm}

\bfseries{\underline{Group members:}}

\begin{enumerate}
        \item Falguni Chakraborty Bsc in IT(Data science)
        \item Sounak Kundu Bsc in IT(AI)
        \item Udita Sarkar BCA
        \item Priyangshu Paul BCA
        \item Mahwish Gul BCA
\end{enumerate}

\vspace{2cm}
 \textbf{Instructor:} Ayan Ghosh \\
 \textbf{Course:} Software Tools And Technology

\newpage

\usetikzlibrary{calc}

\begin{tikzpicture}
[remember picture, overlay] \draw[line width = 2pt] ($(current page.north west) + (0.5in, -0.5in)$) rectangle ($(current page.south east) + (-0.5in, 0.5in)$);
\end{tikzpicture}

\centering

\begin{center}
\Large{\bfseries\underline{Lab Notebook Entries}}
\end{center}

\section{Lab Entry by Falguni Chakraborty}
Reg no- 233002410601
Roll no- 30084323002

\subsection{Experiment}


\begin{table}[ht]
\centering
\begin{tabular}{|p{50pt}|p{200pt}|}
\hline
\textbf{Sl. No.} & \textbf{Assignments} \\ \hline
1. & Introduction to Github and Github desktop version installation \\ \hline
\end{tabular}
\end{table}

\section{Lab Entry by Sounak Kundu}
Reg no- 233002410576
Roll no- 30054623024
\subsection{Experiment}


\begin{table}[ht]
\centering
\begin{tabular}{|p{50pt}|p{200pt}|}
\hline
\textbf{Sl. No.} & \textbf{Assignments} \\ \hline
1. & Building a C programme of calculator in the local repository, committing and publishing it as a public repository. \\ \hline
\end{tabular}
\end{table}

\section{Lab Entry by Mahwish Gul}
Reg no- 233001010529
Roll no - 30001223055
\subsection{Experiment}

\vspace{0.5cm}
\begin{table}[ht]
\centering
\begin{tabular}{|p{50pt}|p{200pt}|}
\hline
\textbf{Sl. No.} & \textbf{Assignments} \\ \hline
1. & Converting submit button to Chin tapak dum dum \\ \hline
\end{tabular}
\end{table}

\newpage

\usetikzlibrary{calc}

\begin{tikzpicture}
[remember picture, overlay] \draw[line width = 2pt] ($(current page.north west) + (0.5in, -0.5in)$) rectangle ($(current page.south east) + (-0.5in, 0.5in)$);
\end{tikzpicture}
\section{Lab Entry by Udita Sarkar}
Reg no: 233001010477
Roll no: 30001223003
\subsection{Experiment}

\vspace{0.5cm}
\begin{table}[ht]
\centering
\begin{tabular}{|p{50pt}|p{200pt}|}
\hline
\textbf{Sl. No.} & \textbf{Assignments} \\ \hline
1. & Introduction to LaTeX \\ \hline
\end{tabular}
\end{table}

\vspace{2.5cm}
\section{Lab Entry by Priyangshu Paul}
Reg no :- 233001010505
Roll No :- 30001223031
\subsection{Experiment}

\vspace{0.5cm}
\begin{table}[ht]
\centering
\begin{tabular}{|p{50pt}|p{200pt}|}
\hline
\textbf{Sl. No.} & \textbf{Assignments} \\ \hline
1. & Creating LaTeX repository on github \\ \hline
\end{tabular}
\end{table}

\newpage

\usetikzlibrary{calc}

\begin{tikzpicture}
[remember picture, overlay] \draw[line width = 2pt] ($(current page.north west) + (0.5in, -0.5in)$) rectangle ($(current page.south east) + (-0.5in, 0.5in)$);
\end{tikzpicture}

\section*{Github}
\paragraph{GitHub is a web-based platform that allows developers to host, share, and collaborate on software projects. It provides a version control system powered by Git, enabling teams to track changes, manage code repositories, and work together seamlessly, even across different locations. GitHub supports collaborative development through features like pull requests, issues, and project boards, making it an essential tool for open-source projects and professional software development alike. Additionally, it offers integration with various development tools, enhancing productivity and streamlining the software development lifecycle.}

\begin{figure}
    \centering
    \includegraphics[width=0.5\linewidth]{GitHub-Symbol.png}
\end{figure}
\subsection*{Installation}
\paragraph{Installing GitHub Desktop is a straightforward process that enhances your workflow by providing a user-friendly interface for managing repositories. To begin, download the installer from the [official GitHub Desktop website](https://desktop.github.com/) for your operating system—Windows or macOS. After downloading, simply run the installer and follow the on-screen instructions to complete the setup. Once installed, you can launch the application and sign in with your GitHub credentials, or create a new account if needed. GitHub Desktop streamlines the process of cloning repositories, making commits, and managing branches, making it an invaluable tool for developers of all skill levels. For Linux users, alternative methods like using Wine or other Git clients are available.}

\end{document}
